\documentclass[12pt]{article}
\usepackage[export]{adjustbox}
\usepackage{amsfonts, amsmath, amssymb}
\usepackage[english, spanish]{babel}
\usepackage{bm}
\usepackage{breqn}
\usepackage{derivative}
\usepackage{empheq}
\usepackage[inline]{enumitem}
\usepackage{fancyhdr}
\usepackage{geometry}
\usepackage{graphicx}
\usepackage{kantlipsum}
\usepackage{lastpage}
\usepackage{linebreaker} % line-breaker algorithm in LuaLaTeX
\usepackage{lualatex-math} % Fixes for mathematics
\usepackage{mathtools}
\usepackage{microtype}
\usepackage{nicefrac}
\usepackage[hyperref]{ntheorem}
\usepackage{phfqit} % Bra-ket notation
% \usepackage{physics}
\usepackage[nolabel]{showlabels}
\usepackage{simples-matrices}
\usepackage{siunitx}
\usepackage{titling}
\usepackage{tabularray}
\usepackage{tasks}
\usepackage{xcolor}
\usepackage{xfrac}
\usepackage{xparse}
\usepackage{xspace}
\usepackage{hyperref}
\usepackage{cleveref}

%%%%%%%%%%%%%%%%%%%%%%%%%%%%%%%%%%%%%%%%%%%%%%%%%%%%%%%%%%
%%%%%%%%%%%%%%%%%%%%%%%%%%%%%%%%%%%%%%%%%%%%%%%%%%%%%%%%%%
%%%%%%%%%%%%%%%%%%Configuraciones extras%%%%%%%%%%%%%%%%%%
\definecolor{base3}{RGB}{253, 246, 227}%
\definecolor{pinkwave}{RGB}{255, 0, 128}%
\definecolor{customBlue}{RGB}{11, 61, 98}%
\pagecolor{base3}
\graphicspath{{img/}}
\setlength{\parindent}{2em} % Sangría
\setlength{\parskip}{0.5em} % Espacio entre párrafos
\linespread{1.1} % line spacing
\setlength{\jot}{10pt} % Space between lines in multiline eqs
\crefname{equation}{ec.}{ecs.} % Equation's cross-reference name

% Line-breaker config
\linebreakersetup {
    maxtolerance = 90,
    maxemergencystretch = 1em,
    maxcycles = 4
}

%%%%%%%%%%%%%%%%%%Theorem environments%%%%%%%%%%%%%%%%%%
% Configuración de ambiente para problema
\theoremstyle{break}
\theoremheaderfont{\Large\normalfont\bfseries}
\theorembodyfont{\normalfont}
\theoremseparator{\bigskip} % Spacing between header and body
\theorempreskip{1.5em}
\theorempostskip{\topsep\bigskip}
\theorempostwork{
    \color{customBlue} \hrule width \hsize height 2pt \kern 1mm \hrule width \hsize
    }
\newtheorem{exercise}{Problema}
% Configuración de ambiente para solución
\theoremstyle{nonumberbreak}
\theoremheaderfont{\Large\normalfont\bfseries}
\theorembodyfont{\normalfont}
\theoremseparator{\medskip}
\theorempreskip{1em}
\theorempostskip{\topsep\medskip}
\newtheorem{solution}{Solución}

%%%%%%%%%%%%%%%%%%%%%%%%%%%%%%%%%%%%%%%%%%%%%%%%%%%%%%%%


% Configuración del paquete hyperref
\hypersetup{
    colorlinks = true,%
    linkcolor={[rgb]{0,0.2,0.6}},%
    citecolor={[rgb]{0,0.6,0.2}},%
    filecolor={[rgb]{0.8,0,0.8}},%
    urlcolor={[rgb]{0.8,0,0.8}},%
    runcolor={[rgb]{0.8,0,0.8}},% 
    menucolor={[rgb]{0,0.2,0.6}},%
    linkbordercolor={[rgb]{0,0.2,0.6}},%
    citebordercolor={[rgb]{0,0.6,0.2}},%
    filebordercolor={[rgb]{0.8,0,0.8}},%
    urlbordercolor={[rgb]{0.8,0,0.8}},%
    runbordercolor={[rgb]{0.8,0,0.8}},%
    menubordercolor={[rgb]{0,0.2,0.6}},% 
    pdftitle={Tarea 1},%
    pdfauthor={López Merino Marcos},%
    pdfsubject={Mecánica Cuántica},%
    pdfkeywords={Facultad de Ciencias, UNAM, Mecánica Cuántica},%
    unicode = true%
}

%%%%%%%%%%%%%%%%%%siunitx configuration%%%%%%%%%%%%%%%%%%
% Configuración del paquete siunitx
\sisetup{
	output-decimal-marker = {.}, 
	per-mode = symbol-or-fraction,
	separate-uncertainty = false,
	exponent-product = \cross,
    % inter-unit-product = \ensuremath{{}\vdot{}}
}

% Declaring new units
\DeclareSIUnit\kilogram{\kilo\gram}

%%%%%%%%%%%%%%%%%%%%%%%%%%%%%%%%%%%%%%%%%%%%%%%%%%%%%%%%

% Geometría del documento
\geometry{
    letterpaper,
    top = 0.6in,
    bottom = 0.8in,
    left = 0.6in,
    right = 0.6in,
    footskip = 38pt
}

%%%%%%%%%%%%%%%%%%Nuevos comandos%%%%%%%%%%%%%%%%%%
\newcommand*{\group}{8449}
\newcommand*{\classname}{Mecánica Cuántica}
\newcommand*{\homeworknumber}{Tarea 1}
\newcommand*{\name}{Marcos López Merino}

% unit vector i
\newcommand*{\uveci}{{\bm{\hat{\textnormal{\bfseries\i}}}}}
% unit vector j
\newcommand*{\uvecj}{{\bm{\hat{\textnormal{\bfseries\j}}}}}
% unit vector
\DeclareRobustCommand{\uvec}[1]{{%
  \ifcsname uvec#1\endcsname
     \csname uvec#1\endcsname
   \else
    \bm{\hat{\mathbf{#1}}}%
   \fi
}}% 
\newcommand{\idest}{\emph{i.e.},\xspace} % id est
% Espacio vectorial, e.g., ℝ, ℂ, ℕ, etc.
\NewDocumentCommand{\vecspace}{m o}{%
  \IfValueTF{#2}{%
    \mathbb{#1}^{#2}%
  }{%
    \mathbb{#1}%
  }%
}

\newcommand*{\e}{\mathrm{e}} % exponential
\newcommand*{\inlinesol}{\vspace*{10pt}\textbf{Solución}\vspace*{10pt}}
\newcommand{\crefrangeconjunction}{--}
\newcommand{\crefpairconjunction}{\xspace y\xspace}

% Defining a variant of Aboxed command from mathtools
\makeatletter
\newcommand*\Acolorboxed[2][pinkwave]{%
   \let\bgroup{\romannumeral-`}%
   \@Acolorboxed{#1}#2&&\ENDDNE
}
\def\@Acolorboxed#1#2&#3&#4\ENDDNE{%
  \ifnum0=`{}\fi
  \setbox\z@\hbox{$\displaystyle#2{}\m@th$\kern\fboxsep \kern\fboxrule}%
  \edef\@tempa{\kern\wd\z@ & \kern-\the\wd\z@ \fboxsep\the\fboxsep \fboxrule\the\fboxrule}%
  \@tempa
  \fcolorbox{#1}{base3}{\m@th$\displaystyle#2#3$}%
} 
\makeatother

%%%%%%%%%%%%%%%%%%Portada y configuración%%%%%%%%%%%%%%%%%%
% Configuración de portada
\setlength{\droptitle}{-60pt} % raise the title

% Portada
\title{
    \textbf{\homeworknumber}\\
    \normalsize\vspace{0.1in}\small{\textbf{Entrega}:~\today}
    \vspace{-1.5in}
}
\author{}
\date{}

%%%%%%%%%%%%%%%%%%Header and footer%%%%%%%%%%%%%%%%%%
\setlength{\headheight}{15.2pt}
\pagestyle{fancy}
\lhead{Grupo \group, Sem. 2023-1}
\chead{\classname}
\rhead{\name}
\lfoot{\includegraphics[scale = 0.2, valign = c]{LogoFCUNAMcolor.pdf}}
% \lfoot{\includegraphics[scale = 0.1, valign = c]{example-image}}
\cfoot{\homeworknumber}
\rfoot{Pág. \thepage \hspace{1pt} de \pageref{LastPage}}

\renewcommand{\headrulewidth}{0.5pt}
\renewcommand{\footrulewidth}{0.5pt}
%%%%%%%%%%%%%%%%%%%%%%%%%%%%%%%%%%%%%%%%%%%%%%%%%%%%%%%%%%
%%%%%%%%%%%%%%%%%%%%%%%%%%%%%%%%%%%%%%%%%%%%%%%%%%%%%%%%%%

\begin{document}
    \maketitle
    \thispagestyle{fancy}
    
    \begin{exercise}
        Resolver los siguientes ejercicios de \emph{Introducción al formalismo de la Mecánica Cuántica no relativista} (Spinel):

        \begin{enumerate}[label = (\alph*)]
            \item Con base en la propiedad (1.1.3), demostrar que el dual de \(c\ket{\beta} = \bra{\beta} c^{*}\).
            
            \inlinesol

            Por el postulado de la correspondencia dual la igualdad se cumple. Es decir, 

            \begin{equation*}
                c\ket{\beta} \xleftrightarrow{\scalebox{0.5}{\text{CD}}} \bra{\beta}c^{*}
            \end{equation*}

            Por lo tanto,

            \begin{empheq}[box = \color{pinkwave}\fbox]{equation*}
                c\ket{\beta} = \bra{\beta}c^{*}
            \end{empheq}
        \end{enumerate}

        Sean \(\ket{\gamma}\) y \(\ket{\eta}\) los ket definidos por: \(\ket{\gamma} = (3 + i)\ket{a_{1}} + 4\ket{a_{2}} - 6i\ket{a_{3}}\) y \(\ket{\eta} = 2i\ket{a_{1}}+ 3\ket{a_{3}}\), donde los kets \(\ket{a_{i}}\) son ortonormales

        \begin{enumerate}[resume*]
            \item Calcule la norma de los kets \(\ket{\gamma}\) y \(\ket{\eta}\) y determine sus kets normalizados \(\ket{\gamma^{\prime}}\) y \(\ket{\eta^{\prime}}\).
            
            \inlinesol

            Puesto que queremos calcular la norma de los kets de \(\ket{\gamma}\) y \(\ket{\eta}\), necesitamos \(\bra{\gamma}\) y \(\bra{\eta}\), tal que,

            \begin{align*}
                \bra{\gamma} &= (3 - i)\bra{a_{1}} + 4\bra{a_{2}} + 6i\ket{a_{3}},\\
                \bra{\eta} &= -2i\bra{a_{1}} + 3\bra{a_{3}}\text{.}
            \end{align*}

            Así,

            \begin{align}
                \sqrt{\braket{\gamma}{\gamma}} &= \left\lbrace[(3 - i)\bra{a_{1}} + 4\bra{a_{2}} + 6i\ket{a_{3}}][(3 + i)\ket{a_{1}} + 4\ket{a_{2}} - 6i\ket{a_{3}}]\right\rbrace^{\sfrac{1}{2}},\nonumber\\
                &= \left\lbrace(3 - i)(3 + i) + 4(4) + 6i(-6i)\right\rbrace^{\sfrac{1}{2}},\nonumber\\
                \Acolorboxed{\sqrt{\braket{\gamma}{\gamma}} &= \sqrt{62}}\label{eq:GammaNorm}
            \end{align}

            Ahora calculamos \(\sqrt{\braket{\eta}{\eta}}\),

            \begin{align}
                \sqrt{\braket{\eta}{\eta}} &= \left\lbrace[-2i\bra{a_{1}} + 3\bra{a_{3}}][2i\ket{a_{1}}+ 3\ket{a_{3}}]\right\rbrace^{\sfrac{1}{2}},\nonumber\\
                &= \left\lbrace(-2i)(2i) + 3(3)\right\rbrace^{\sfrac{1}{2}},\nonumber\\
                \Acolorboxed{\sqrt{\braket{\eta}{\eta}} &= \sqrt{13}}\label{eq:EtaNorm}
            \end{align}

            Para normalizar un ket debemos dividirlo por su norma, \textbf{i.e.}, debemos dividir \(\ket{\gamma}\) y \(\ket{\eta}\) por las \cref{eq:GammaNorm,eq:EtaNorm}, respectivamente.

            \begin{empheq}[box = \color{pinkwave}\fbox]{align}
                \ket{\gamma^{\prime}} &= \dfrac{1}{\sqrt{62}}[(3 + i)\ket{a_{1}} + 4\ket{a_{2}} - 6i\ket{a_{3}}],\label{eq:GammaNormalized}\\
                \ket{\eta^{\prime}} &= \dfrac{1}{\sqrt{13}}[2i\ket{a_{1}}+ 3\ket{a_{3}}]\text{.}\label{eq:EtaNormalized}
            \end{empheq}

            \item Encuentre los bras correspondientes a los kets \(\ket{\gamma^{\prime}}\) y \(\ket{\eta^{\prime}}\).
            
            \inlinesol

            Puesto que las \cref{eq:GammaNormalized,eq:EtaNormalized} son los kets \(\ket{\gamma}\) y \(\ket{\eta}\) normalizados, respectivamente; entonces, sus bras correspondientes son \(\sfrac{1}{\sqrt{62}}\bra{\gamma}\) y \(\sfrac{1}{\sqrt{13}}\bra{\eta}\). Así,

            \begin{empheq}[box = \color{pinkwave}\fbox]{align*}
                \bra{\gamma^{\prime}} &= \frac{1}{\sqrt{62}}[(3 - i)\bra{a_{1}} + 4\bra{a_{2}} + 6i\ket{a_{3}}],\\
                \bra{\eta^{\prime}} &= \frac{1}{\sqrt{13}}[-2i\bra{a_{1}} + 3\bra{a_{3}}]\text{.}
            \end{empheq}

            \item Calcule el producto interior \(\braket{\gamma^{\prime}}{\eta^{\prime}}\) y demuestre por cálculo directo que es igual a \(\left\lbrace\braket{\eta^{\prime}}{\gamma^{\prime}}\right\rbrace^{*}\).
            
            \inlinesol

            Calculamos \(\braket{\gamma^{\prime}}{\eta^{\prime}}\) y \(\braket{\eta^{\prime}}{\gamma^{\prime}}\).

            \begin{itemize}[label = \textbullet]
                \item \(\braket{\gamma^{\prime}}{\eta^{\prime}}\)
                
                \begin{align}
                    \braket{\gamma^{\prime}}{\eta^{\prime}} &= \frac{1}{\sqrt{806}}[(3 - i)\bra{a_{1}} + 4\bra{a_{2}} + 6i\ket{a_{3}}][-2i\bra{a_{1}} + 3\bra{a_{3}}],\nonumber\\
                    \Acolorboxed{\braket{\gamma^{\prime}}{\eta^{\prime}} &= \frac{1}{\sqrt{806}}(2 + 24i)}\label{eq:GammaprimeEtaprime}
                \end{align}

                \item \(\braket{\eta^{\prime}}{\gamma^{\prime}}\)
                
                \begin{align}
                    \braket{\eta^{\prime}}{\gamma^{\prime}} &= \frac{1}{\sqrt{806}}[-2i\bra{a_{1}} + 3\bra{a_{3}}][(3 - i)\bra{a_{1}} + 4\bra{a_{2}} + 6i\ket{a_{3}}],\nonumber\\
                    \Acolorboxed{\braket{\gamma^{\prime}}{\eta^{\prime}} &= \frac{1}{\sqrt{806}}(2 - 24i)}\label{eq:EtaprimeGammaprime}
                \end{align}

                Calculando el conjugado de la \cref{eq:EtaprimeGammaprime},

                \begin{empheq}[box = \color{pinkwave}\fbox]{equation}
                    \left\lbrace\braket{\eta^{\prime}}{\gamma^{\prime}}\right\rbrace^{*} = \frac{1}{\sqrt{806}}(2 + 24i)
                    \label{eq:EtaprimeGammaprimeConjugate}
                \end{empheq}
            \end{itemize}

            Por lo tanto, de las \cref{eq:GammaprimeEtaprime,eq:EtaprimeGammaprimeConjugate},

            \begin{empheq}[box = \color{pinkwave}\fbox]{equation*}
                \braket{\gamma^{\prime}}{\eta^{\prime}} = \left\lbrace\braket{\eta^{\prime}}{\gamma^{\prime}}\right\rbrace^{*}
            \end{empheq}

            \item Calcules los productos interiores \(\braket{a_{1}}{\eta^{\prime}},\ \braket{a_{2}}{\eta^{\prime}}\) y \(\braket{a_{3}}{\eta^{\prime}}\). De acuerdo con sus resultados ¿qué interpretación geométrica puede dar al producto interior?
            
            \inlinesol

            Calculamos \(\braket{a_{1}}{\eta^{\prime}},\ \braket{a_{2}}{\eta^{\prime}}\) y \(\braket{a_{3}}{\eta^{\prime}}\).


            \begin{empheq}[box = \color{pinkwave}\fbox]{align*}
                \braket{a_{1}}{\eta^{\prime}} &= \frac{1}{\sqrt{62}}(3 + i),\\
                \braket{a_{2}}{\eta^{\prime}} &= \frac{4}{\sqrt{62}},\\
                \braket{a_{3}}{\eta^{\prime}} &= \frac{-6i}{\sqrt{62}}\text{.}
            \end{empheq}

            De las expresiones anteriores podemos determinar que el producto interior nos da la proyección de \(\eta^{\prime}\) sobre \(a_{1},\ a_{2}\) y \(a_{3}\), respectivamente.
        \end{enumerate}
    \end{exercise}

    \pagebreak
    \begin{exercise}
        \begin{enumerate}[label = (\alph*)]
            \item Find the condition under which two vectors
            
            \begin{equation*}
                \ket{v_{1}} = \matrice[1]{x, y, 3}, \quad \ket{v_{2}} = \matrice[1]{2, x - y, 1} \in \vecspace{R}[3]
            \end{equation*}

            are linearly independent.

            \inlinesol

            Sabemos que si dos vectores son linealmente dependientes podemos expresar uno de estos como un múltiplo del otro, \textbf{i.e.}, 

            \begin{equation*}
                \ket{v_{1}} = c\ket{v_{2}}\text{.}
            \end{equation*}

            Sustituyendo \(\ket{v_{1}}\) y \(\ket{v_{2}}\) en la expresión anterior:

            \begin{equation*}
                \matrice[1]{x, y, 3} = c\matrice[1]{2, x - y, 1} = \matrice[1]{2c, c(x - y), c}\text{.}
            \end{equation*}

            Así,

            \begin{align}
                x &= 2c,\label{eq:2a-firsteq}\\
                y &= c(x - y),\label{eq:2a-secondeq}\\
                c &= 3\text{.}\label{eq:2a-thirdeq}
            \end{align}

            Sustituimos la \cref{eq:2a-thirdeq} en la \cref{eq:2a-firsteq},

            \begin{align}
                x &= 2(3),\nonumber\\
                x &= 6\text{.}\label{eq:2a-xvalue}
            \end{align}

            Ahora, simplificamos la \cref{eq:2a-secondeq} y sustituimos las \cref{eq:2a-thirdeq,eq:2a-xvalue}, tal que,

            \begin{align*}
                y(1 + c) &= cx,\\
                y(1 + 3) &= 18,\\
                y = \frac{9}{2}\text{.}
            \end{align*}

            Por lo tanto, las condiciones para que los vectores sean l.i. son:

            \begin{empheq}[box = \color{pinkwave}\fbox]{align*}
                x &= 6,\\
                y &= \frac{9}{2},\\
                c &= 3\text{.}
            \end{empheq}

            \item Show that a set of vectors
            
            \begin{equation*}
                \ket{v_{1}} = \matrice[1]{1, 1, 1},\quad \ket{v_{2}} = \matrice[1]{1, 0, 1}, \quad \ket{v_{3}} = \matrice[1]{1, -1, -1}
            \end{equation*}

            is a basis of \(\vecspace{C}[3]\).

            \inlinesol

            Para determinar si el conjunto de vectores \(\left\lbrace\ket{v_{1}}, \ket{v_{2}}, \ket{v_{3}}\right\rbrace\) es una base de \(\vecspace{C}[3]\) únicamente asta calcular el determinante de la matriz generada por los vectores y, verificar que éste si es diferente de cero, en caso de serlo, los vectores son linealmente independientes y, por lo tanto, son base.

            \begin{align*}
                \det M &= \det\matrice[3]{1, 1, 1, 1, 0, -1, 1, 1, -1},\\
                \det M &= 2 \neq 0\text{.} 
            \end{align*}

            Por lo tanto,
            
            \begin{empheq}[box = \color{pinkwave}]{equation*}
                \left\lbrace\ket{v_{1}}, \ket{v_{2}}, \ket{v_{3}}\right\rbrace\, \text{es una base de}\, \vecspace{C}[3]\text{.}
            \end{empheq}

            \pagebreak
            \item Let
            
            \begin{equation*}
                \ket{x} = \matrice[1]{1, i, 2 + i}, \quad
                \ket{y} = \matrice[1]{2 - i, 1, 2 + i}\text{.}
            \end{equation*}

            Find \(\norm{\ket{x}}\), \(\innerprod{x}{y}\) and \(\innerprod{y}{x}\).

            \inlinesol

            \begin{itemize}[label = \textbullet]
                \item Norma
                
                Primero calculamos \(\braket{x}{x}\), tal que,

                \begin{align*}
                    \braket{x}{x} &= \matrice[3]{1, -i, 2 - i}\matrice[1]{1, i, 2+ i},\\
                    \braket{x}{x} &= 7\text{.}
                \end{align*}

                Aplicando la raíz cuadrada a la expresión anterior, tenemos que la norma es igual a

                \begin{empheq}[box = \color{pinkwave}\fbox]{equation*}
                    \sqrt{\braket{x}{x}} = \sqrt{7}\text{.}
                \end{empheq}

                \item \(\braket{x}{y}\)
                
                \begin{align*}
                    \braket{x}{y} &= \matrice[3]{1, -i, 2 - i}\matrice[1]{2 - i, 1, 2 + i},\\
                    \Acolorboxed{\braket{x}{y} &= 7 - 2i}
                \end{align*}

                \item \(\braket{y}{x}\)
                
                \begin{align*}
                    \braket{y}{x} &= \matrice[3]{2 + i, 1, 2 - i}\matrice[1]{1, i, 2 + i},\\
                    \Acolorboxed{\braket{y}{x} &= 7 + 2i}
                \end{align*}
            \end{itemize}

            \pagebreak
            \item
            
            \begin{enumerate}[label = \arabic*.]
                \item Use the Gram-Schmidt orthonormalization to find an orthonormal basis \(\left\lbrace \ket{e_{k}}\right\rbrace\) from a linearly independent set of vectors
                
                \begin{equation*}
                    \ket{v_{1}} = \matrice[3]{-1{,}, 2{,}, 2}^{t},
                    \quad
                    \ket{v_{2}} = \matrice[3]{2{,}, -1{,}, 2}^{t},
                    \quad
                    \ket{v_{3}} = \matrice[3]{3{,}, 0{,}, -3}^{t}
                \end{equation*}

                \inlinesol

                Primero definimos un vector, para normalizamos \(\ket{v_{1}}\), tal que,

                \begin{equation*}
                    \ket{e_{1}} = \dfrac{\ket{v_{1}}}{\sqrt{\braket{v_{1}}{v_{1}}}},
                \end{equation*}

                donde la norma es

                \begin{equation*}
                    \sqrt{\braket{v_{1}}{v_{1}}} = 3\text{.}
                \end{equation*}

                Entonces, 

                \begin{empheq}[box = \color{pinkwave}\fbox]{equation}
                    \ket{e_{1}} = \dfrac{\matrice[3]{-1, 2, 2}^{t}}{3}\text{.}
                    \label{eq:2d-e1}
                \end{empheq}

                Ahora definimos el vector \(\ket{f_{2}}\),

                \begin{align*}
                    \ket{f_{2}} &= \matrice[3]{2, -1, 2}^{t} - \frac{1}{3}\matrice[1]{-1, 2, 2}\matrice[3]{-1, 2, 2}\matrice[1]{2, -1, 2},\\
                    \ket{f_{2}} &= \matrice[3]{2, -1, 2}^{t}\text{.}
                \end{align*}

                Por otro lado, tenemos que \(\sqrt{\braket{f_{2}}{f_{2}}} = 3\). Entonces el vector \(\ket{e_{2}}\) está dado como:
            
                \begin{empheq}[box = \color{pinkwave}\fbox]{equation}
                    \ket{e_{2}} = \dfrac{\matrice[3]{2, -1, 2}^{t}}{3}\text{.}
                    \label{eq:2d-e2}
                \end{empheq}

                Definimos el vector \(\ket{f_{3}}\), tal que,

                \begin{align*}
                    \ket{f_{3}} &= \ket{v_{3}} - \braket{e_{1}}{v_{3}}\ket{e_{1}} - \braket{e_{2}}{v_{3}}\ket{e_{2}},\\
                    \ket{f_{3}} &= \matrice[3]{2, 2, -1}^{t}\text{.}
                \end{align*}

                Tenemos que \(\sqrt{\braket{f_{3}}{f_{3}}} = 3\), entonces el vector \(\ket{e_{3}}\) es

                \begin{empheq}[box = \color{pinkwave}\fbox]{equation}
                    \ket{e_{3}} = \dfrac{\matrice[3]{2, 2, -1}^{t}}{3}
                \end{empheq}

                \item Let
                
                \begin{equation*}
                    \ket{u} = \matrice[3]{1{,}, -2{,}, 7}^{t} =
                    \sum_{k} c_{k}\ket{e_{k}}
                \end{equation*}

                Find the coefficients \(c_{k}\).

                \inlinesol

                Reescribimos \(\ket{u}\), tal que,

                \begin{equation*}
                    \ket{u} = \matrice[1]{\frac{-1}{3}c_{1} + \frac{2}{3}c_{2} + \frac{2}{3}c_{3}, \frac{2}{3}c_{1} - \frac{1}{3}c_{2} + \frac{2}{3}c_{3}, \frac{2}{3}c_{1} + \frac{2}{3}c_{2} - \frac{1}{3}c_{3}} = \matrice[1]{1, -2, 7}\text{.}
                \end{equation*}

                Resolviendo el sistema de ecuaciones generado por la expresión anterior, tenemos que el valor de los coeficientes es:

                \begin{empheq}[box = \color{pinkwave}\fbox]{align*}
                    c_{1} &= 3,\\
                    c_{2} &= 6,\\
                    c_{3} &= -3\text{.}
                \end{empheq}
            \end{enumerate}

            % \pagebreak
            \item Let
                
                \begin{equation*}
                    \ket{v_{1}} = \matrice[3]{1{,}, i{,}, 1}^{t},
                    \quad
                    \ket{v_{2}} = \matrice[3]{3{,}, 1{,}, i}^{t}\text{.}
                \end{equation*}

                Find the orthonormal basis for a two-dimensional subspace spanned by \(\left\lbrace\ket{v_{1}}, \ket{v_{2}}\right\rbrace\).

                \inlinesol

                Para encontrar una base ortonormal para un subespacio dimensional, lo hacemos usando la ortonormalización de Gram-Schmidt.
                
                Primero definimos el vector \(\ket{e_{1}}\), donde \(\sqrt{\braket{v_{1}}{v_{1}}} = \sqrt{3}\).

                \begin{empheq}[box = \color{pinkwave}\fbox]{equation}
                    \ket{e_{1}} = \frac{1}{\sqrt{3}}\matrice[1]{1, i, 1}
                    \label{eq:2e-e1}
                \end{empheq}

                Ahora definimos el vector \(\ket{f_{2}}\),

                \begin{align*}
                    \ket{f_{2}} &= \ket{v_{2}} - \braket{e_{1}}{v_{2}}\ket{e_{1}},\\
                    \ket{f_{2}} &= \matrice[3]{2, 1 - i, i - 1}^{t}\text{.}
                \end{align*}

                Para normalizar \(\ket{f_{2}}\), primero debemos calcular su norma, tal que,
                
                \begin{equation*}
                    \sqrt{\braket{f_{2}}{f_{2}}} = 2\sqrt{2}\text{.}
                \end{equation*}

                Entonces, \(\ket{e_{2}}\) queda como

                \begin{empheq}[box = \color{pinkwave}\fbox]{equation*}
                    \ket{e_{2}} = \frac{1}{2\sqrt{2}}\matrice[1]{2, 1 - i, i - 1}\text{.}
                \end{empheq}
        \end{enumerate}
    \end{exercise}

    \pagebreak
    \begin{exercise}
        \begin{enumerate}[label = (\alph*)]
            \item Let \(x \neq 0\) and \(y \neq 0\). \begin{enumerate*}[label = (\alph*)]
                \item If \(x \perp y\), show that \(\left\lbrace x, y\right\rbrace\) is a linearly independent set.
                \item Extend the result to mutually orthogonal nonzero vectors \(x_{1}, \dots, x_{m}\).
            \end{enumerate*}

            \item Let \(z_{1}\) and \(z_{2}\) denote complex numbers. Show that \(\innerprod{z_{1}}{z_{2}} = z_{1}\bar{z}_{2}\) defines an inner product, which yields the usual metric on the complex plane. Under what condition do we have orthogonality?
            
            \item Show that the norm on \(C[a, b]\) is invariant under a linear transformation \(t = \alpha\tau + \beta\). Use this to prove that the statement in 3.1-8 by mapping \([a, b]\) onto \([0, 1]\) and then considering the functions defined by \(\bar{x}(\tau) = 1,\ \bar{y}(\tau) = \tau\), where \(\tau\in[0, 1]\).
            
            \item If \(X\) is a finite dimensional vector space and \((e_{j})\) is a basis for \(X\), show that an inner product on \(X\) is completely determined by its values \(\gamma_{jk} = \innerprod{e_{j}}{e_{k}}\). Can we choose such scalar \(\gamma_{jk}\) in a completely arbitrary fashion?
            
            \item If \((e_{k})\) is an orthonormal sequence in an inner product space \(X\), and \(x \in X\), show that \(x - y\) with \(y\) given by
            
            \begin{equation*}
                y = \sum_{k = 1}^{n} \alpha_{k}e_{k}
                \condition*{\alpha_{k} = \innerprod{x}{e_{k}}}
            \end{equation*}

            is orthogonal to the subspace \(Y_{n} = \textrm{span}\left\lbrace e_{1}, \dots, \e_{n}\right\rbrace\).

            \item Orthonormalize the first three terms of the sequence \((x_{0}, x_{1}, x_{2}, \cdots)\), where \(x_{j}(t) = t^{j}\), on the interval \([-1, 1]\), where
            
            \begin{equation*}
                \innerprod{x}{y} = \int_{-1}^{1} x(t)y(t)\odif{t}.
            \end{equation*}
        \end{enumerate}
    \end{exercise}
\end{document}